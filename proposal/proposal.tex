\documentclass{article}
\usepackage{natbib}
\usepackage{parskip}

\title{State Space Graph Mining}
\author{Bryan Silverthorn\\
\texttt{bsilvert@cs.utexas.edu}
\and Craig Corcoran\\
\texttt{ccor@cs.utexas.edu}}

\begin{document}
\maketitle

%
% document body
%

\section*{Project Proposal}

The engineering of an appropriate state space representation is one of the
essential prerequisites to applying reinforcement learning (RL) in a new
domain. The choice of representation is often critical to the success of an RL
agent, just as feature engineering is often critical to the success of a
supervised learning method. This representation must be constructed by hand,
and requires an understanding of the peculariaties of a given RL method.

Many domains, however, are naturally represented as a discrete state space
graph: the game of Tic-Tac-Toe, for example, is easily described as a set of
boards (3x3 grids), with the possible moves linking boards with directed edges.
This type of representation is verbose, and not directly amenable to
well-studied standard RL methods, but may serve as a useful starting point---an
initial representation, to which data mining tools can be applied to extract
useful lower-dimensionality state descriptions.

Recent work \citep{Wang2009Multiscale,Mahadevan2006Value} has begun to examine
the use of graph analysis methods in this framework, focusing on diffusion
wavelets and the eigenvectors of the graph Laplacian, but important questions
remain: how can this approach be scaled to large state spaces? Can it be made
to work given only sampled walks through the state space (e.g., data from human
gameplay)? Which methods generate state descriptions suitable for which RL
algorithms?

Our project will tackle these questions in applying data mining tools to
automatically derive state representations for two complex, large-scale
domains: the board game of Go, using state trajectories extracted from expert
human games; and the operation of a modern satisfiability (SAT) solver,
\texttt{MiniSat}, on standard benchmark collections.

%
% references
%

\bibliographystyle{abbrvnat}
\bibliography{references}

\end{document}

