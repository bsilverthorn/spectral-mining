\section{Future Work}

This paper has laid down the outline of a general approach to representation
discovery in board games. Many areas of this outline, however, deserve to be
further refined and explored. Broadly, three areas of possible future work seem
most promising.

First, this paper has evaluated only the most straightforward vector-space
mapping for constructing the affinity graph. Using existing, domain-specific
heuristic features to measure state similarity could improve the quality of
the resulting graph. Similarly, there is an opportunity to use learning to 
optimize the affinity space mapping.

Second, alternative approaches to scaling could be explored. Whether
partitioning should occur on a graph representation or in the initial vector
space, for example, is an open question. An efficient approximate method for
constructing the nearest-neighbor graph would also be useful. Additionally, the computational and performance tradeoffs between number of 
clusters, number of eigenvectors solved for per cluster, and sample size should
be further explored.

Third, value-function approximation may be most useful working alongside
existing effective Monte Carlo methods for Go: the value function can provide
high-level strategic insight gleaned from expert examples, as a complement to
the precise tactical judgments of MCTS. An interesting avenue for future work
would develop such a hybrid.

Overall, the direction of future work is toward refining 
the feature discovery pipeline outlined in this paper, and toward maximizing the
value of applying it. 

