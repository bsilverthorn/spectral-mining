\section{Conclusion}

This work has extended spectral representation discovery to board-game domains,
which pose unique challenges, both of kind and of scale, not present in the
small spatial domains employed in existing work. Three ideas are proposed:
approximating the full gameplay graph using coarse state similarity; operating
on partitions of the graph obtained from clustering; and extracting information
about the space of relevant games from expert examples. Experiments cover the
games of Tic-Tac-Toe and Go. The features learned in these experiments are, for
example, sufficiently informative to represent an effective policy for
Tic-Tac-Toe, and to substantially improve value function prediction over
baseline features in Go. Future work will further develop these methods.

