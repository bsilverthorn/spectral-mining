\section{Introduction}

The field of reinforcement learning (RL) concerns itself with situations in
which an agent learns from interactions with its environment. Value-based
approaches to RL estimate the expected utility of each possible state of the
agent and its environment---the so-called \emph{value function}---and use that
value function to produce a \emph{policy} that maps from states to actions.

Just as feature engineering is often critical to the success of supervised
learning methods, the success of value-based RL depends on how well the value
function is represented. This work will explore new methods for automatically
generating such a representation, with a focus on approaches that scale to
large discrete domains.

\subsection{Linear Value Function Approximation}

In environments where the number of states is large, an agent using a value
function must employ some form of approximation. A linear architecture is often
chosen for simplicity and ease of analysis. In such a representation, the value
$V(s)$ of a state $s$ is estimated as a linear combination of features
$\phi(s)$ of that state:
%
\begin{equation}
V(s) = w^{T}\phi(s)
\end{equation}
%
As in any linear prediction scheme, the quality of the resulting value function
and the performance of the resulting policy depends critically on the basis
functions or state features $\phi(s)$. These features are typically constructed
by hand, a labor-intensive process that requires extensive domain knowledge.

Prior work in automatically generating representations for value function
approximation include population-based search methods
\citep{whiteson2006evolutionary}, methods based on Bellman error
\citep{parr2007analyzing}, and methods based on augmented Krylov techniques
\citep{petrik2007analysis}, as well as many others. This work builds upon a
particularly promising approach that leverages results from the harmonic
analysis of graphs to aid in discovering useful representations
\citep{Mahadevan2006Value,Coifman06Diffusion,Wang2009Multiscale}.

\subsection{Spectral Representation Discovery}

Spectral graph theory provides many useful tools for analyzing a graph and
functions on that graph \citep{Chung1997Spectral}. The graph Laplacian $L$ and
its normalized variants play a central role in this analysis. Given the
adjacency matrix of a weighted graph $W$, the graph Laplacian is defined as
%
\begin{equation}
L = D-W
\end{equation}
%
where $D$ is defined as the row sums of $W$:
%
\begin{equation}
D = \sum_{i} W_{ij}
\end{equation}
%
The normalized graph Laplacian is defined to be:
%
\begin{equation}
\mathcal{L} = D^{-1/2}LD^{-1/2}
\label{eqn:norm.laplacian}
\end{equation}
%
The graph Laplacian has a number of useful properties; among others, its
eigenvectors form a good basis for representing smooth functions on the graph.

The state space of a finite Markov decision process (MDP) can be represented
as a graph where each vertex is a state and edges represent state transitions.
Using this state adjacency graph $W$, we can form the (normalized) graph 
Laplacian and use the eigenvectors as a basis for performing linear value 
function approximation.

Recent work
\citep{petrik2007analysis,mahadevan2006learning,Mahadevan2006Value,mahadevan2007proto}
has begun to examine the use of harmonic analysis of graphs in this framework,
but have focused on domains small enough for direct application of approaches
similar to those described above. As the state space grows, however,
representing the full adjacency graph and finding its eigenvectors becomes
prohibitively expensive. Some prior research has applied Kronecker matrix
factorization to the problem of scaling \citep{johns2007compact}. Our work will
instead move toward an approach based on clustering the state graph.

We focus on two domains in this preliminary submission: first, a simple
two-room grid world that highlights the desirable properties of the Laplacian
eigenvectors in terms of representing the topology of a state space; and
second, a classic game, Tic-Tac-Toe (TTT), that serves as an easily-understood
domain in which to develop and evaluate techniques for applying spectral
methods to the unusual state graphs of board games. Both domains lead us toward
future work on the larger, more challenging board game of Go.

