\documentclass{article}
\usepackage{natbib}
\usepackage{parskip}

\title{Discovering Representations\\by Mining State-Space Graphs}
\author{Bryan Silverthorn\\
\texttt{bsilvert@cs.utexas.edu}
\and Craig Corcoran\\
\texttt{ccor@cs.utexas.edu}}

\begin{document}
\maketitle

%
% document body
%

\section*{Project Proposal}

The engineering of an appropriate state space representation is one of the
essential prerequisites to applying reinforcement learning (RL) in a new
domain. The choice of representation is often critical to the success of an RL
agent, just as feature engineering is often critical to the success of a
supervised learning method. This representation is usually constructed by hand,
and requires an understanding of the peculiarities of a given domain.

Many domains, however, are naturally represented as a discrete state space
graph: the game of Tic-Tac-Toe, for example, is easily described as a set of
boards (3x3 grids), with the possible moves linking boards with directed edges.
This type of representation is verbose, and not directly amenable to
well-studied standard RL methods, but may serve as a useful starting point---an
initial representation, to which data mining tools can be applied to extract
useful lower-dimensionality state descriptions.

Recent work \citep{Wang2009Multiscale,Mahadevan2006Value,Coifman06Diffusion} has 
begun to examine the use of harmonic analysis of graphs in this framework, 
focusing on diffusion wavelets and the eigenvectors of the graph Laplacian, but 
important questions remain: how can this approach be scaled to large state spaces? 
Can it be made to work given only sampled walks through the state space (e.g., data 
from human gameplay)? Which methods generate state descriptions suitable for which 
RL algorithms?

To address scalability, one possible research direction would be to apply a
multilevel graph clustering algorithm, such as graclus
\citep{Dhillon07weightedgraph}, then perform harmonic state space analysis on
the clusters separately or on a coarsened representation of the state graph. It
may also be worthwhile to use the clusters directly as binary functions; many
RL algorithms become simple and efficient in the binary-feature case.  

Our project will tackle these questions in applying data mining tools to
automatically derive state representations for two complex, large-scale
domains: the board game of Go, using state trajectories extracted from expert
human games; and the operation of a modern satisfiability (SAT) solver,
\texttt{MiniSat}, on standard benchmark collections. The success of our
approach will be measured against current state-of-the-art handcrafted template
features \citep{Silver2007Shape} in the case of Go, and against default solver
configurations in the SAT domain. 

\section*{Tentative Schedule}

\begin{itemize}
\item 10/14: Demonstrate learning in the tic-tac-toe (TTT) domain using the
eigenvectors of the graph Laplacian as basis functions.
\item 10/21: Demonstrate learning in the 5x5 Go domain, again using the
eigenvectors of the graph Laplacian. Consider how to generalize outside the
state space graph. Analyze performance as the number of basis functions varies.
\item 10/28: Implement and compare strategies for handling the state space size
in Go at larger board sizes, focusing on graph coarsening approaches.
\item 11/4: Explore the approaches that appear promising from results collected
so far. Implement a proof-of-concept application to SAT.
\item 11/11--: Continue to iterate.
\end{itemize}

%
% references
%

\bibliographystyle{abbrvnat}
\bibliography{references}

\end{document}

