\section{Future Work}

\begin{figure}
\begin{center}
\includegraphics[width=\textwidth]{results/ttt_graph.sampled.pdf}
\end{center}
\caption{\label{fig:ttt.sampled.graph}An incomplete TTT gameplay graph
constructed from sampled games, i.e., from repeated stochastic walks over the
complete gameplay graph of \cref{fig:ttt.gameplay.graph}. Each vertex
represents a board configuration, with the empty starting board at the center
of the graph, and each edge represents a move. Green circles denote positions
in which player 1 is to move, as do blue squares for player 2. Every path from
the root to a leaf represents a complete possible Tic-Tac-Toe game. While the
tree-like structure is clear, the complete gameplay graph also includes many
nodes in which multiple game paths intersect, i.e., it remains a DAG. This
structural complexity will largely disappear in \cref{sec:scaling}, when only
sampled paths are available.}
\end{figure}

This work moves toward the case where we do not have direct access to the full
state space, only sample trajectories. As \cref{fig:ttt.sampled.graph}
illustrates, these samples are often disjoint, forming long isolated chains
that may not provide a sufficiently rich topology to apply spectral
representation-discovery techniques directly. We also work toward approaches
that can be applied when finding eigenvalues is computationally infeasible,
even on the sampled Laplacian matrix. We show that clustering as a
preprocessing step may be a viable remedy to both problems.

