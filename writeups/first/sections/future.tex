\section{Future Work}

\begin{figure}
\begin{center}
\includegraphics[width=\textwidth]{results/ttt_graph.sampled.pdf}
\end{center}
\caption{\label{fig:ttt.sampled.graph}An incomplete TTT gameplay graph
constructed from sampled games, i.e., from repeated stochastic walks over the
complete gameplay graph of \cref{fig:ttt.gameplay.graph}. Although some edges
cross between game trajectories, most are isolated, and the rich topology of
the game is not evident. This problem is likely to be even more pronounced in
samples from a more complex game. If future work is to apply spectral analysis
to sampled game trajectories, more informed graph construction must be
performed.}
\end{figure}

The number of game states in most board games increases combinatorially with
the size of the board. The set of all possible Go board configurations, for
example, becomes intractably large even on only a small 9-by-9 board.

This work therefore moves toward the case where we do not have direct access to
the full state space, and instead must consider only sample trajectories of
games already played. These trajectories are informative because they indicate
important regions of the game space, but as \cref{fig:ttt.sampled.graph}
illustrates, they are often also disjoint; they form long, isolated chains that
may not provide a sufficiently rich topology to apply spectral
representation-discovery techniques directly. State affinity information may
allow us to join these chains---perhaps by aliasing states with an existing
clustering method.

We must also work toward approaches to apply when finding eigenvalues is
computationally infeasible even on the sampled Laplacian matrix. Clustering as
a preprocessing step to reduce the size of the graph may be a viable remedy to
this problem as well.

For both statistical and computational reasons, then, the development and use
of effective state aliasing could be an important ingredient in scaling to more
challenging domains. It will be our primary focus in future work.

