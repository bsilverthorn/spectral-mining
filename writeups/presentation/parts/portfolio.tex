\section{Algorithm Portfolios}

\begin{frame}
In practice, humans do this learning all the time.
\end{frame}

\begin{frame}
``Algorithm A isn't working; let me try algorithm B.''
\end{frame}

\begin{frame}
What about automating such reasoning?
\end{frame}

\begin{frame}
Great idea! You've invented an algorithm portfolio.
\end{frame}

\begin{frame}
Let's say we know each algorithm's probabilities of success:

\begin{center}
\begin{tabular}{lllll}
\toprule
& \multicolumn{4}{l}{{\bf Budget (CPU Seconds)}}\\
\cmidrule{2-5}
{\bf Algorithm} & {\bf 30 s} & {\bf 60 s} & {\bf 90 s} & {\bf 120 s}\\
\midrule
Solver A & $0.10$ & $0.20$ & $0.60$ & $0.65$\\
Solver B & $0.25$ & $0.26$ & $0.27$ & $0.28$\\
\bottomrule
\end{tabular}
\end{center}

How do we maximize \emph{our} probability of success?
\end{frame}

\begin{frame}

What is our best action with $N$ seconds remaining?

It's the action that minimizes our joint probability of failure:
\[ (1 - p(\textrm{success}\ |\ \textrm{solver}_i, \textrm{budget}_i)) * (1 - p(\textrm{success}\ |\ \textrm{best}_{N - \textrm{budget}_i})) \]

% XXX the above equation is reasonably confusing

\end{frame}

\begin{frame}
What is our \textcolor{darkblue}{best} action with $30$ seconds remaining?

\begin{center}
\begin{tabular}{lllll}
\toprule
& \multicolumn{4}{l}{Success Probability After}\\
\cmidrule{2-5}
Algorithm & 30 s & 60 s & 90 s & 120 s\\
\midrule
Solver A & $0.10$ & $0.20$ & $0.60$ & $0.65$\\
Solver B & \textcolor{darkblue}{$0.25$} & $0.26$ & $0.27$ & $0.28$\\
\textcolor{darkblue}{Best Plan} & 0.25 & ? & ? & ?\\
\bottomrule
\end{tabular}
\end{center}

\dots B for 30, with failure probability $0.75$.
\end{frame}

\begin{frame}
What is our \textcolor{darkblue}{best} action with $60$ seconds remaining?

\begin{center}
\begin{tabular}{lllll}
\toprule
& \multicolumn{4}{l}{Success Probability After}\\
\cmidrule{2-5}
Algorithm & 30 s & 60 s & 90 s & 120 s\\
\midrule
Solver A & $0.10$ & $0.20$ & $0.60$ & $0.65$\\
Solver B & \textcolor{darkblue}{$0.25$} & $0.26$ & $0.27$ & $0.28$\\
\textcolor{darkblue}{Best Plan} & $0.25$ & $0.44$ & ? & ?\\
\bottomrule
\end{tabular}
\end{center}

\dots B for 30, again, with failure probability $0.75 * 0.75 = 0.56$.
\end{frame}

\begin{frame}
What is our \textcolor{darkblue}{best} action with $90$ seconds remaining?

\begin{center}
\begin{tabular}{lllll}
\toprule
& \multicolumn{4}{l}{Success Probability After}\\
\cmidrule{2-5}
Algorithm & 30 s & 60 s & 90 s & 120 s\\
\midrule
Solver A & $0.10$ & $0.20$ & \textcolor{darkblue}{$0.60$} & $0.65$\\
Solver B & $0.25$ & $0.26$ & $0.27$ & $0.28$\\
\textcolor{darkblue}{Best Plan} & $0.25$ & $0.44$ & $0.60$ & ?\\
\bottomrule
\end{tabular}
\end{center}

\dots A for 90, with failure probability $0.40$.
\end{frame}

\begin{frame}
What is our \textcolor{darkblue}{best} action with 120 seconds remaining?

\begin{center}
\begin{tabular}{lllll}
\toprule
& \multicolumn{4}{l}{Success Probability After}\\
\cmidrule{2-5}
Algorithm & 30 s & 60 s & 90 s & 120 s\\
\midrule
Solver A & $0.10$ & $0.20$ & \textcolor{darkblue}{$0.60$} & $0.65$\\
Solver B & $\textcolor{darkblue}{0.25}$ & $0.26$ & $0.27$ & $0.28$\\
\textcolor{darkblue}{Best Plan} & $0.25$ & $0.44$ & $0.60$ & $0.70$\\
\bottomrule
\end{tabular}
\end{center}

\dots A for 90, again, with failure probability $0.40 * 0.75 = 0.30$.
\end{frame}

\begin{frame}
\begin{textblock*}{5in}(0pt,0pt)
% XXX figure removed
%\input{figures/portfolio_oracle.tikz.tex}
\end{textblock*}
\end{frame}

\begin{frame}
Of course, for a new instance, we \emph{don't} know those probabilities!
\end{frame}

\begin{frame}
---and we can only know them \emph{with certainty} by solving the instance.
\end{frame}

\begin{frame}
We do know the history of run costs for each solver,

\begin{center}
\begin{tabular}{llll}
\toprule
Instance & Solver & TTS (s)\\
\midrule
X & A & \textcolor{black}{99}\\
  &   & \textcolor{black}{22}\\
  &   & \textcolor{black}{103}\\
Y &   & \textcolor{black}{116}\\
  &   & \textcolor{black}{17}\\
  &   & \textcolor{black}{8}\\
\vdots & \vdots & \vdots\\
\bottomrule
\end{tabular}
\end{center}
\end{frame}

\begin{frame}
and we can cluster each of these solver histories:

\begin{center}
\begin{tabular}{lll}
\toprule
Instance & Solver & TTS (s)\\
\midrule
X & A & \textcolor{darkred}{99}\\
  &   & \textcolor{darkblue}{22}\\
  &   & \textcolor{darkred}{103}\\
Y &   & \textcolor{darkred}{116}\\
  &   & \textcolor{darkblue}{17}\\
  &   & \textcolor{darkblue}{8}\\
\vdots & \vdots & \vdots\\
\bottomrule
\end{tabular}
\end{center}
\end{frame}

\begin{frame}
These clusters give us probabilities in multiple possible ``worlds'':

\begin{center}
\begin{tabular}{llllll}
\toprule
& & \multicolumn{4}{l}{Success Probability After}\\
\cmidrule{3-6}
Algorithm & Cluster & 30 s & 60 s & 90 s & 120 s\\
\midrule
Solver A & \textcolor{darkblue}{\#0} & $0.90$ & $0.98$ & $1.00$ & $1.00$\\
& \textcolor{darkred}{\#1} & $0.02$ & $0.05$ & $0.13$ & $0.94$\\
\vdots & \vdots & \vdots & \vdots & \vdots & \vdots\\
\bottomrule
\end{tabular}
\end{center}
\end{frame}

\begin{frame}
We can then group the run clusters by \emph{task}, and cluster again:
\ \\
\ \\
\begin{center}
\begin{tabular}{llll}
\toprule
& & \multicolumn{2}{l}{Run Class Prob.}\\
\cmidrule{3-4}
Task Class & Solver & \textcolor{darkblue}{RC \#0} & \textcolor{darkred}{RC \#1}\\
\midrule
TC \#0 & A & $0.91$ & $0.09$\\
       & B & $0.11$ & $0.89$\\
TC \#1 & A & $0.03$ & $0.97$\\
       & B & $0.84$ & $0.16$\\
\vdots & \vdots & \vdots & \vdots\\
\bottomrule
\end{tabular}
\end{center}
\ \\
\ \\
Run failures narrow down both the \emph{task cluster} and the \emph{run clusters}.
\end{frame}

\begin{frame}
What if we have static information about a new instance?
\ \\
\ \\
\begin{center}
\begin{tabular}{llll}
\toprule
Task & Clauses & Balance & \dots\\
\midrule
X & 3000 & 0.22 & \dots\\
Y & 400 & 0.90 & \dots\\
\vdots & \vdots & \vdots &\\
\bottomrule
\end{tabular}
\end{center}
\end{frame}

\begin{frame}
We can train a classifier to predict the task cluster probabilities.
\ \\
\ \\
\begin{center}
\begin{tabular}{llllll}
\toprule
Task & Clauses & Balance & \dots & TC \#0 & TC \#1\\
\midrule
X & 3000 & 0.22 & \dots & 0.12 & 0.88\\
Y & 400 & 0.90 & \dots & 0.70 & 0.30\\
\vdots & \vdots & \vdots & \dots & \vdots & \vdots\\
Z & 500 & 0.87 & \dots & \textcolor{darkgreen}{0.74?} & \textcolor{darkgreen}{0.26?}\\
\bottomrule
\end{tabular}
\end{center}
\end{frame}

\begin{frame}
%One obvious inefficiency: we can only \emph{stop} and \emph{restart} solvers.
What if a solver run \emph{would have} succeeded, given an extra few seconds?

Let's permit solver runs to be resumed.
\end{frame}

\begin{frame}
Augment our probability matrices to include paused solver runs.
\ \\
\ \\
\begin{center}
\begin{tabular}{lllll}
\toprule
& \multicolumn{4}{l}{{\bf Budget (CPU Seconds)}}\\
\cmidrule{2-5}
{\bf Algorithm} & {\bf 30 s} & {\bf 60 s} & {\bf 90 s} & {\bf 120 s}\\
\midrule
Solver A & $0.10$ & $0.20$ & $0.60$ & $0.65$\\
Solver B & $0.25$ & $0.26$ & $0.27$ & $0.28$\\
Solver A@30 & $0.20$ & $0.60$ & $0.65$ & $0.65$\\
\bottomrule
\end{tabular}
\end{center}
\end{frame}

\begin{frame}
\begin{textblock*}{5in}(0pt,0pt)
% XXX removed figure
%\input{figures/portfolio_all.tikz.tex}
\end{textblock*}
\end{frame}

\begin{frame}
\begin{textblock*}{5in}(0pt,0pt)
% XXX removed figure
%\input{figures/portfolio_all_sat07.tikz.tex}
\end{textblock*}
\end{frame}

\begin{frame}
What if we have multiple cores?

Repeatedly re-plan, assuming that in-progress actions will fail.
\end{frame}

