\section{Introduction}

\begin{frame}
Consider solving an instance of satisfiability (SAT), e.g.:
\begin{center}
\tikz \node [rectangle,draw]
{ $(x_0 \vee \neg x_2 \vee \cdots) \wedge (\neg x_1 \vee x_2 \vee \cdots) \wedge \cdots$ };
\end{center}
It's typical to consider only \emph{that instance}.
\end{frame}

\begin{frame}
---but no instance is an island.

\begin{center}
\tikz \node [rectangle,draw]
{ $(x_0 \vee \neg x_2 \vee \cdots) \wedge (\neg x_1 \vee x_2 \vee \cdots) \wedge \cdots$ };
\tikz \node [rectangle,draw]
{ $(x_0 \vee x_2 \vee \cdots) \wedge (\neg x_1 \vee \neg x_2 \vee \cdots) \wedge \cdots$ };
\tikz \node [rectangle,draw]
{ $(\neg x_0 \vee \neg x_2 \vee \cdots) \wedge (x_1 \vee x_2 \vee \cdots) \wedge \cdots$ };
\\ $\vdots$
\end{center}

Instances are different, yet they are often still related.
\end{frame}

\begin{frame}
Can we \emph{learn} to solve problems more efficiently?

Can the problems we've \textcolor{darkgreen}{solved} help with those we \textcolor{darkred}{haven't}?

\begin{center}
\tikz \node [rectangle,draw,darkgreen]
{ \textcolor{black}{$(x_0 \vee \neg x_2 \vee \cdots) \wedge \cdots$} };
\tikz \node [rectangle,draw,darkgreen]
{ \textcolor{black}{$(x_0 \vee x_2 \vee \cdots) \wedge \cdots$} };
\tikz \node [rectangle,draw,darkgreen]
{ \textcolor{black}{$(\neg x_0 \vee \neg x_2 \vee \cdots) \wedge \cdots$} };

\tikz \draw [->] (0,0) -- (0,-0.5);

Learning!

\tikz \draw [->] (0,0) -- (0,-0.5);

\tikz \node [rectangle,draw,darkred]
{ \textcolor{black}{$(x_0 \vee \neg x_2 \vee \cdots) \wedge \cdots$} };
\tikz \node [rectangle,draw,darkred]
{ \textcolor{black}{$(x_0 \vee x_2 \vee \cdots) \wedge \cdots$} };
\tikz \node [rectangle,draw,darkred]
{ \textcolor{black}{$(\neg x_0 \vee \neg x_2 \vee \cdots) \wedge \cdots$} };
\end{center}
\end{frame}

