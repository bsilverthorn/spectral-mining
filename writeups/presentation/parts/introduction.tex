\section{Introduction}

\begin{frame}
\frametitle{Problem Description}

\begin{itemize}
  \item Model board game as an MDP
  \item We want to estimate the value function:
    \begin{align*}
        V^{*}(x) &= \max_{a \in \mathcal{A}_x} [ R(x) + \gamma V(T(x, a)) ]\\
        V(s)     &= w^{T}\phi(s)
    \end{align*}
  \item Good $V(s) \rightarrow$ Good greedy policy
  \item Generate $\phi(s)$ automatically
\end{itemize}
\end{frame}

\begin{frame}
\frametitle{How to Generate Features?}
Mine the state-space graph
(TODO - state space diagram here)
\end{frame}

\begin{frame}
\frametitle{What to do with the graph?}
\begin{itemize}
  \item Weighted Adjacency Matrix $W$
  \item Form Graph Laplacian: $L = D-W $
  \item take $k$ "smallest" eigenvectors: $Lv = \lambda v$
    \begin{itemize}
        \item As in Spectral Clustering
        \item Good basis for smooth functions on graph $\rightarrow$ features 
    \end{itemize}
\end{itemize}
\end{frame}

\begin{frame}
\frametitle{TicTacToe Eigenvectors}
\includegraphics[width=\textwidth]{figures/ttt_move_evs}
\end{frame}

\begin{frame}
\frametitle{Affinity Graph for Large Domains}
In large games using full state-space graph is intractable.
\begin{itemize}
  \item Sample using recorded expert games
    \begin{itemize}
    \item reveals relevant region of state space
    \end{itemize}
  \item form K-NN graph in board space $\rightarrow W$
\end{itemize} 
\ \\
\begin{equation}
W_{ij} = \exp(-\frac{\vectornorm{x_{i}-x_{j}}^2}{2 \sigma^2})
\end{equation}
\end{frame}

\begin{frame}
\frametitle{TTT Value-Function Prediction Error}
\includegraphics[width=\textwidth]{figures/ttt_prediction}
\end{frame}

\begin{frame}
\frametitle{Go Results} 
\includegraphics[width=\textwidth]{figures/go_prediction}
\end{frame}

\begin{frame}
\frametitle{Demo!} 
(TODO - Demo URL)
\end{frame}


\begin{frame}
\frametitle{Future Work} 
Obstacles to scaling
    \begin{itemize}
    \item Large number of samples
            \begin{itemize}
            \item Constructing K-NN graph
            \item Eigenvalue Computation
            \end{itemize} 
    \end{itemize} 

Work in Progress
    \begin{itemize}
    \item Using k-means to cluster samples
            \begin{itemize}
            \item Perform feature generation on subgraphs
            \end{itemize}
    \item Use a better affinity space representation
            \begin{itemize}
            \item Use hand-crafted, symmetry-invariant features
                \begin{itemize}
                \item Feature amplification
                \end{itemize}
            \item Learning
            \end{itemize}
    \end{itemize}
\end{frame}
